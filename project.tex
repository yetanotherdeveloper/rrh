%& --translate-file il2-t1
\documentclass[a4paper,10pt]{article}
%\usepackage[polish]{babel}
%\usepackage{polski}
\usepackage{makeidx}
%\usepackage[utf-8]{inputenc}
%\usepackage[T1]{fontenc}
\usepackage[a4paper,left=2.5cm,right=2.0cm,top=1.5cm,bottom=1.5cm]{geometry}
% Title Page
\title{Notatki dotyczace prob robienia filmow}
\author{Jacek Czaja}
\makeindex

\begin{document}

\maketitle

\tableofcontents

\section{Plan dzialan}
\begin{enumerate}
\item Tutorial/screecast o image morphing
\item Obejrzec tutoriale (VSE, NOD editor)
\end{enumerate}


\section{Technologia}
\subsection{Sekwencja klatek}
Zrzucamy zdjecia, najlepiej jak sa ponumerowane. Przykladowa linia polecen by zlozyc je w film (codec x264):
\begin{verbatim}
ffmpeg -start_number 334 -i DSC00%d.JPG  -map 0 -c:v libx264 -c:a copy tiger.mp4
\end{verbatim}

\subsection{Spowolnienie filmu}
efekt "slow motion" mozna uzyskac za pomoca oprogramowania slowmoVideo.
Os Y to jest odtwarzanie wejsciowego filmu , a krzywa ktora ustalamy mapuje Y na X gdzie X to
odtwarzanie wyjsciowego obrazu.\\
Shutter function to jest okreslenie jka duzo blura nalezy zrobic to jest symulacja czasu naswietlania. 

\subsection{Podglad strumienia z gopro}
Przy kolejnym wlaczeniu podlgadu, moze zajac troche czasu zanim sie obraz pojawi
\begin{verbatim}
mplayer ffmpeg://http://10.5.5.9:8080/live/amba.m3u8
\end{verbatim}

\subsection{Miecz swietlny}
Komenda by skryptem colorglow zrobic miecz switlny:
\begin{verbatim}
-c "133,78" -s 100 -e 8 -f 20 -h -20
\end{verbatim}

\subsection{Rysowanie po ekranie}
\begin{enumerate}
\item Rozbic film na dwie czesci
\item wyciagnac klatke na ktorej bedziemy rysowac
\item Uzywajac gimpa narysowac malunek
\item Robic undo i kazdy przejsciowy obrazek zapisac
\item zlozyc obrazki razem z dwoma czesciami filmu uzywajac blendra 
\end{enumerate}
\subsubsection{Bezstratne przekodowanie filmu}
\begin{verbatim}
ffmpeg -i 00000.MTS -vf scale=1280:800 -c:v libx264 -preset ultrafast -qp 0 -c:a copy output_fast.mkv
\end{verbatim}

\subsubsection{Najazd kamery na obrazek}
Pojedyncze zdjecie badz sekwencje zdjec wczytac do blendera (plugin image as a plane). Nastepnie Import->Image as Plane
Ustawic kamere w pozycji poczatkowej i nastepnie w koncowej (zmieniajac klatke z pierwszej na ostatnia).
W ustawieniach obiektu (kemery) mozna klawiszem "i" zapamietac lokalizacje kamery tzn zapamietac w pierwzej klatce poczatkowa
i w ostatniej klatce koncowa (ewentaulnie jakies posrednie tez. 
Oswietlenie nalezy ustawic w world (ambient occlusion).

\subsubsection{Renderowanie animacji w blenderze z linii polecen}
\begin{verbatim}
~/blender-2.76b-linux-glibc211-x86_64/blender -b miasteczko-keying.blend -a
\end{verbatim}

\subsection{Green screen}
\begin{itemize}
\item Keying node posiada przylacze: core oraz garbage, pozwala podlaczyc maske ktore
Kluczowanie mozna robic w blenderze. 
\item Tworzymy maske i dodajemy wierzcholki a pozniej "i" zeby dodac klatki kluczowe  
\item Maske mozna zrobic odbic lustrzanie za pomoc flip node
\item Keying node sluzy do kluczowania po kolarze
\end{itemize}


\subsubsection{Zasady dobrego greenscreena}
\begin{itemize}
\item Zachowac odleglosc obiektu kluczowanego od green screena by nie bylo cienii
\item Oswietlenie bialy jednorodne (swiatlo dzienne alebo biale oswietlenie z lamp)  
\item Bez zagniecen materialu
\end{itemize}

\subsubsection{Wyciagniecie jednej klatki z danego momentu}
\begin{verbatim}
ffmpeg -i output_fast.mkv -ss 00:00:5.000 -f image2 -vframes 1 out.png
\end{verbatim}

\subsection{Morphing}
Dwie klatki (zrodlowa i docelowa) ustawiamy w blender z pomoca image as plane. Docelowa minimalnie za zrodlowa.
Pozniej dodajemy shape key dla obu obrazkow. Nastepnie robimy dla obu subdivide i pozniej modyfikujemy
wierzcholki za pomoca connected mesh. Pozniej orotognal projection. Oraz modyfikacja alpha materialow obu obrazkow
Robmy zrodlowy shape key by zmienial sie w docelowy oraz docelowy w zrodlowy. Jak tylko zrodlowy w docelowy to
bedzie cross fading a nie morphing

\subsection{wybuchy, flary}
Kulisty obiekt, z teksura cloud, ktora nie renderujemy tylko uzywamy jako
displace map. Pozniej weight painting zeby ksztalt byl odpowiedni
nastepnie w materiale: volume na zadany kolor.
Density na 0 gdy ma nie byc efektu

\subsection{Usuniecie szumu z nagrania dzwiekowego}
\begin{enumerate}
\item Wytnij kawalek szumu z nagrania: \textsf{sox nagranie.wav szum.wav trim start duration}
\item Stworz profil szumu: \textsf{sox szum.wav -n noiseprof noise.prof} 
\item Usun szum z nagrania: \textsf{sox nagranie.wav nagranie-bez-szumu.wav noisered noise.prof 0.21}
\end{enumerate}

\subsection{Obrot filmu o 90 stopni}
\subsubsection{zgodnie ze wskazowkami zegara}
\begin{verbatim}
ffmpeg -i in.mov -vf "transpose=1" out.mov
\end{verbatim}
\subsubsection{przeciwnie do wskazowek zegara}
\begin{verbatim}
ffmpeg -i in.mov -vf "transpose=2" out.mov
\end{verbatim}

\subsection{Zwiekszenie ilosc klatek na sekunde poparzez duplikacje/wyciecie klatek}
\begin{verbatim}
 ffmpeg -i <input_file> -r 29.97 <output_file>
\end{verbatim}

\subsubsection{Wyciagniecie fragmentu filmu usuwajac audio}
\begin{verbatim}
ffmpeg -i output_fast2.mkv -ss 00:00:00 -t 00:00:5.000 -an before.mkv
\end{verbatim}

\subsection{Informacje}
\begin{itemize}
\item Aby na andrroidzie ogladac wyrendorowane z linuxa filmy to format i codec w blenderze ustawilem na mpeg
\item odtwarzanie mplayerem z zatrzymaniem na ostatniej klatce:
\begin{verbatim}
mplayer -idle -fixed-vo <movie file>
\end{verbatim}
\end{itemize}

\section{TODO}
\begin{itemize}
\item Przetestowac skrypt do robienia mieczy swietlnych za pomoca imagemagic
\item Efekt wybuchu zrobic by na filmie sie pojawil
\end{itemize}

\end{document}

